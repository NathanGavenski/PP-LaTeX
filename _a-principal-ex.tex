%----------------------------------------------------------------------------------
% Exemplo do uso da classe pucrs-ppgcc.cls.
%----------------------------------------------------------------------------------

% Seleção de idioma da monografia. Por enquanto as únicas opções
% suportadas são 'portuguese' e 'english'
% Para impressão em frente e verso, use a opção 'twoside'. Da
% mesma forma, use 'oneside' para impressão em um lado apenas.
\documentclass[portuguese,oneside]{pucrs-ppgcc}
%\documentclass[english,oneside]{pucrs-ppgcc}

%---------------------------------------------------------
% 05/04/2021 - Para utilizar o formato de referências APALIKE:
% por ex.: (Nagamochi and Abe, 2007)
% Fazer uso do pacote 'natbib'. Descomentar a linha 70.
%----------------------------------------------------------------
% Coloque seus pacotes abaixo.
%
% Obs.: muitos pacotes de uso comum do LaTeX, como amsmath,
% geometry e url já são automaticamente incluídos pela classe
% (veja o arquivo .cls). Isso torna obrigatória a presença destes
% no sistema para o uso desta classe, mas ao mesmo tempo o uso se
% torna mais simples.  Recomendo a instalação da versão mais
% recente da distribuição TeXLive (para Windows e UNIXes):
% www.tug.org/texlive/
%
% Pacotes e opções já incluídas automaticamente:
%
% \RequirePackage[T1]{fontenc}[2005/09/27]
% \RequirePackage[utf8x]{inputenc}[2008/03/30]
% \RequirePackage[english,brazil]{babel}[2008/07/06]
% \RequirePackage[a4paper]{geometry}[2010/09/12]
% \RequirePackage{textcomp}[2005/09/27]
% \RequirePackage{lmodern}[2009/10/30]
% \RequirePackage{indentfirst}[1995/11/23]
% \RequirePackage{setspace}[2000/12/01]
% \RequirePackage{textcase}[2004/10/07]
% \RequirePackage{float}[2001/11/08]
% \RequirePackage{amsmath}[2000/07/18]
% \RequirePackage{amssymb}[2009/06/22]
% \RequirePackage{amsfonts}[2009/06/22]
% \RequirePackage{url}
% \RequirePackage[table]{xcolor}[2007/01/21]
%\RequirePackage{array}[2008/09/09]
%\RequirePackage{longtable}
%----------------------------------------------------------------
% Para inserção de figuras.
\usepackage{graphicx}
% Utilize a opção 'pdftex' se você estiver usando o pdflatex (que
% permite figuras em formatos como .jpg ou .png)
%\usepackage[pdftex]{graphicx}

% Para tabelas com elementos ocupando mais de uma linha
\usepackage{multirow}
% Para frações na mesma linha (ex. ⅓).
\usepackage{nicefrac}
% Para inserir figuras lado a lado.
% \usepackage{subfigure}
% Para formatar algoritmos.
% A opção [algo2e] é necessária para evitar conflitos
% com as definições da classe.
%\usepackage[algo2e]{algorithm2e}
\usepackage{algorithmic}
% Um float do tipo algoritmo. No momento
% este pacote é incompatível com a classe.
%\usepackage{algorithm}

%----------------------------------------------------------------
% Utilizar o pacote 'natbib', SOMENTE se fizer uso do estilo APALIKE:
%\usepackage{natbib}

%----------------------------------------------------------------
% Autor (OBRIGATÓRIO)
%----------------------------------------------------------------
\author{Discente da Silva}

%----------------------------------------------------------------
% Título (OBRIGATÓRIO). Devem ser passados DOIS parâmetros,
% o título em português E o inglês, não importando o idioma
% escolhido. Os títulos são utilizados para a montagem da capa,
% resumo e abstract mais tarde.
%----------------------------------------------------------------
\title{Seu título em português aqui}
      {Your title in english here}

%----------------------------------------------------------------
% Opções para o tipo de trabalho (OBRIGATÓRIO)
%----------------------------------------------------------------
%\tipotrabalho{\monografia}  % Monografias em geral (e de "bônus": TCCs)
%\tipotrabalho{\pep}         % Plano de estudo e pesquisa (Research plan)
\tipotrabalho{\dissertacao} % Dissertação (Master Thesis)
%\tipotrabalho{\ptese}       % Proposta de tese (Doctoral Thesis proposal)
%\tipotrabalho{\tese}        % Tese (Doctoral Thesis)

%----------------------------------------------------------------
% Seleção do curso ("este trabalho é um requisito parcial para
% obtenção do grau de (mestre(a) ou doutor(a)) em Ciência da Computação").
% Necessário somente para o tipo 'monografia'.
%----------------------------------------------------------------
%\grau{\bacharela} % Este é "bônus"
%\grau{\bacharel} % Este é "bônus"
\grau{\mestra}
%\grau{\mestre}
%\grau{\doutora}
%\grau{\doutor}

%----------------------------------------------------------------
% Informar o gênero do Orientador(a) e Co-orientador(a) 
% Apenas um para cada.
%----------------------------------------------------------------
%\generoori{\orientador}
\generoori{\orientadora}
%\generocoori{\coorientador}
\generocoori{\coorientadora}

%----------------------------------------------------------------
% Orientador(a) (e Co-orientador(a), caso haja um/uma). 
% É OBRIGATÓRIO informar pelo menos o(a) orientador(a).
%----------------------------------------------------------------
\orientador{Beltrano(a) Dias}
\coorientador{Ciclano(a) de Farias}

%----------------------------------------------------------------
% A capa é inserida automaticamente. Por isso não é necessário
% chamar \maketitle
%----------------------------------------------------------------
\begin{document}

%----------------------------------------------------------------
% Depois da capa vem a dedicatória e a epígrafe.
%----------------------------------------------------------------
\dedicatoria{Dedico este trabalho a meus pais.}

\epigrafe{The art of simplicity is a puzzle of complexity.}
         {Douglas Horton}

%----------------------------------------------------------------
% Também dá para fazer as duas na mesma página:
%----------------------------------------------------------------
%\dedigrafe{Dedico este trabalho a meus pais.}
%          {The art of simplicity is a puzzle of complexity.}
%          {Douglas Horton}

%----------------------------------------------------------------
% A seguir, a página de agradecimentos (OPCIONAL):
%----------------------------------------------------------------
\begin{agradecimentos}
À lorem ipsum, dolor sit amet consetetur sadipscing elitr sed diam
nonumy eirmod tempor. invidunt ut labore et dolore magna aliquyam

À erad sed, diam voluptua at vero, eos et accusam et justo duo
dolores et ea rebum stet clita.

À kasd gubergren, no sea. takimata sanctus est lorem ipsum dolor sit
amet lorem ipsum dolor sit amet. consetetur sadipscing elitr sed

À diam nonumy, eirmod tempor, invidunt ut labore et dolore magna
aliquyam erat sed diam voluptua at.
\end{agradecimentos}

%----------------------------------------------------------------
% Resumo, com as palavras-chave passadas por parâmetro
% (OBRIGATÓRIO, ao menos para teses e dissertações)
%----------------------------------------------------------------
\begin{resumo}{lorem, ipsum, dolor, sit, amet}
Seu resumo em português aqui. lorem ipsum dolor sit amet
consetetur sadipscing elitr sed diam nonumy eirmod tempor invidunt
ut labore et dolore magna aliquyam erat sed diam voluptua at vero
eos et accusam et justo duo dolores et ea rebum stet clita.  kasd
gubergren no sea takimata sanctus est lorem ipsum dolor sit amet
lorem ipsum dolor sit amet consetetur sadipscing elitr sed diam
nonumy eirmod tempor invidunt ut labore et dolore magna aliquyam
erat sed diam voluptua at.
\end{resumo}

%----------------------------------------------------------------
% Abstract, com as palavras-chave passadas por parâmetro
% (OBRIGATÓRIO, ao menos para teses e dissertações)
%----------------------------------------------------------------
\begin{abstract}{lorem, ipsum, dolor, sit, amet}
Your abstract in English here. lorem ipsum dolor sit amet
consetetur sadipscing elitr sed diam nonumy eirmod tempor invidunt
ut labore et dolore magna aliquyam erat sed diam voluptua at vero
eos et accusam et justo duo dolores et ea rebum stet clita kasd
gubergren no sea takimata sanctus est lorem ipsum dolor sit amet
lorem ipsum dolor sit amet consetetur sadipscing elitr sed diam
nonumy eirmod tempor invidunt ut labore et dolore magna aliquyam
erat sed diam voluptua at
\end{abstract}

%----------------------------------------------------------------
% Listas e sumário, nessa ordem. Somente o sumário é obrigatório,
% portanto, comente as outras listas, caso sejam desnecessárias.
%----------------------------------------------------------------
\listoffigures       % Lista de figuras      (OPCIONAL)
\listoftables        % Lista de tabelas      (OPCIONAL)
\listofalgorithms    % Lista de algoritmos   (OPCIONAL)
\listofacronyms      % Lista de siglas       (OPCIONAL)
\listofabbreviations % Lista de abreviaturas (OPCIONAL)
\listofsymbols       % Lista de símbolos     (OPCIONAL)
\tableofcontents     % Sumário               (OBRIGATÓRIO)

%----------------------------------------------------------------
% Aqui começa o desenvolvimento do trabalho. Para uma melhor
% organização do documento, separe-o em arquivos,
% um para cada capítulo. Para isso, utilize o comando \include,
% como mostrado abaixo.
%----------------------------------------------------------------
%!TEX root = _a-principal-ex.tex
% Acima, um exemplo de como colocar o arquivo principal para compilar de
% qualquer arquivo.
%----------------------------------------------------------------------------------
% Exemplo do uso da classe pucrs-ppgcc.cls.
%----------------------------------------------------------------------------------
\chapter{\label{chap:intro}Introdução}
% Comando para inserir siglas. Tanto as siglas quanto as
% abreviaturas devem aparecer em ordem ALFABÉTICA nas listas
% correspondentes. Como a classe no momento não é capaz de ordenar
% as entradas automaticamente, existem duas alternativas:
%
%    a- Insira todas as siglas e abreviaturas no começo do texto,
%    manualmente e em ordem alfabética.
%
%    b- Caso esteja em um ambiente UNIX (Linux, Mac ou Cygwin/similares),
%    utilize o script sort.sh e o makefile que acompanham a
%    classe. O makefile automaticamente compila a monografia para
%    PDF (mas assume que o latex está acessível pela linha de
%    comando). Neste caso a ordenação é feita de forma automática.
%
\sigla{abc}{aaaaa bbbbb ccccc}
\sigla{xyz}{lorem ipsum dolor sit}
\sigla{ijk}{lorem ipsum dolor sit}
%
% Comando para inserir abreviaturas.
%
\abrev{Abrev}{Abreviatura}
\abrev{Inform}{Informática}
%
% Comando para inserir símbolos. Eles irão aparecer em ordem
% de ocorrência, já que o número da página está presente na lista
% de símbolos.
\simbolo{Hz}{Hertz}
\simbolo{$\pi$}{Constante com valor aproximado de $3.1415926$}%

lorem ipsum dolor sit amet Capítulo~\ref{chap:intro} consetetur
sadipscing elitr sed diam nonumy eirmod tempor invidunt ut labore
et dolore magna aliquyam erat sed diam voluptua at vero eos et
accusam et justo duo dolores et ea rebum stet clita kasd gubergren
no sea takimata sanctus est lorem ipsum dolor sit amet lorem ipsum
dolor sit amet consetetur sadipscing elitr sed diam nonumy eirmod.
Ver Figura~\ref{fig:fig1}.

% Um exemplo de figura
\begin{figure}[htb!]
\centering\includegraphics[width=.65\textwidth]{fig/exemplo}
\caption%[This figure has a shorter caption now]%
        {\label{fig:fig1}This is a figure with a very long
    caption which looks ugly in the corresponding list of figures.
    Fortunately, there is an optional parameter for a shorter
    replacement of this monstrosity}%
\end{figure}

tempor invidunt~\cite{SKIENAC698} ut labore et dolore magna
aliquyam erat sed diam voluptua at vero eos et accusam et justo
duo dolores et ea rebum stet clita kasd gubergren no sea takimata
sanctus est lorem ipsum dolor sit amet lorem
ipsum~\cite{NAGAPACKING07}. O Algoritmo~\ref{alg:alg1}
mostra este processo.

\begin{algorithm}[htb]
\begin{center}
    % Um exemplo de algoritmo utilizando a pacote 'algorithmic'
    %\algsetup{linenosize=\small,linenodelimiter=.}
    \begin{algorithmic}[1]
        \STATE \textbf{function} $\sigma\left(i,j\right)$
        \STATE \COMMENT{\texttt{table} lorem ipsum dolor consetetur sadipscing elitr sed $\left(i,j\right)$}
        \IF{$\text{table}\left[i,j\right].\text{memoized}$}
            \RETURN $\text{table}\left[i,j\right].\text{error}$
        \ENDIF
        \STATE $\text{minerror}\leftarrow\infty$
        \STATE $\text{bestt}\leftarrow{}\text{nil}$
        \FOR { each template $t$ in $T$ }
            \STATE $\text{error}\leftarrow{}\text{allocate}\left(t,i,j\right)$
            \IF{$\text{error}<\text{minerror}$}
                \STATE $\text{minerror}\leftarrow{}\text{error}$
                \STATE $\text{bestt}\leftarrow{}t$
            \ENDIF
        \ENDFOR
        \STATE $\text{table}\left[i,j\right].\text{memoized}\leftarrow{}\text{true}$
        \STATE $\text{table}\left[i,j\right].\text{template}\leftarrow{}\text{bestt}$
        \STATE $\text{table}\left[i,j\right].\text{error}\leftarrow{}\text{minerror}$
        \RETURN $\text{minerror}$
    \end{algorithmic}
\end{center}
\caption[An algorithm with an optional, shorter caption]%
    {\label{alg:alg1}This is an algorithm with a very long
    caption. However, we replaced it with a shorter version
    in the Outline for legibility reasons}%
\end{algorithm}

tempor invidunt ut labore et dolore magna aliquyam erat sed diam
voluptua at vero eos et accusam et justo duo dolores et ea
rebum~\cite{CORMEMALGORITHMS01}.

dolor sit~\cite{BENTLEYBC07} amet consetetur sadipscing elitr sed
diam nonumy eirmod tempor invidunt ut labore et dolore magna
aliquyam erat sed diam voluptua at vero~\cite{BRIANPL04}.

\begin{enumerate}
   \item lorem
   \item ipsum
   \item dolor
   \item sit
   \item amet
   \item consetetur
\end{enumerate}

\section{\label{sec:secao1}Primeira seção}

lorem ipsum dolor sit $x\leq 2$  amet consetetur sadipscing elitr
sed diam nonumy eirmod Seção~\ref{sec:secao1} tempor invidunt ut
labore et dolore magna aliquyam erat sed diam voluptua at vero eos
et accusam et justo duo dolores et ea rebum stet
clita.~\cite{OLIVEIRAAPL08}

% Um exemplo arbitrário de fórmula
\begin{equation}\label{eq:eq1}
    \intop_{0}^{\infty}{x^2 + \frac{\pi}{\sum_{i=0}^{n}{\frac{1}{i^2}}}}
\end{equation}

kasd gubergren no sea Equação~\eqref{eq:eq1} takimata sanctus est
lorem ipsum dolor sit amet lorem ipsum dolor sit amet consetetur
sadipscing elitr sed diam nonumy eirmod.~\cite{PICCOLIAPL11}

amet lorem ipsum dolor sit amet consetetur sadipscing elitr sed
diam nonumy eirmod.~\cite{PICCOLIDM08}

\subsection{Subseção}

dolor sit amet consetetur sadipscing elitr sed diam nonumy eirmod
tempor invidunt ut labore et dolore magna aliquyam erat sed diam
voluptua at vero.

lorem ipsum dolor sit amet consetetur sadipscing elitr sed diam
nonumy eirmod tempor invidunt ut labore et dolore magna aliquyam
erat sed diam voluptua at vero eos et accusam et justo duo dolores
et ea rebum stet clita kasd gubergren no sea takimata sanctus est
lorem ipsum dolor sit amet lorem ipsum dolor sit amet consetetur
sadipscing elitr sed diam nonumy eirmod.

tempor invidunt ut labore et dolore magna aliquyam erat sed diam voluptua at
vero eos et accusam et justo duo dolores et ea rebum stet clita kasd
gubergren no sea takimata sanctus est lorem ipsum dolor sit amet lorem ipsum
dolor sit amet consetetur sadipscing elitr sed diam nonumy eirmod tempor
invidunt ut labore et dolore magna aliquyam erat sed diam voluptua
at vero:

\begin{itemize}
   \item lorem
   \item ipsum
   \item dolor
   \item sit
   \item amet
   \item consetetur
\end{itemize}

\subsubsection{Subsub}

lorem ipsum dolor sit amet consetetur sadipscing elitr sed diam nonumy
eirmod tempor invidunt ut labore et dolore magna aliquyam erat sed diam
voluptua at vero eos et accusam et justo duo dolores et ea rebum
stet clita.~\cite{GOLDENBERGAPL02}

\begin{equation}\label{eq:apl}%
    L\left(i,j,w,h\right)=
    \begin{cases}
      E\left(i,w,h\right) & i=j\\
      \min\left(\min_{k=i}^{j-1}\left\{\heartsuit\left(i,k,j,w,h\right)\right\},
                \min_{k=i}^{j-1}\left\{\spadesuit\left(i,k,j,w,h\right)\right\}\right) & i<j\text{.}
    \end{cases}
\end{equation}

lorem ipsum dolor sit amet consetetur sadipscing elitr sed diam nonumy
eirmod tempor invidunt ut labore et dolore magna aliquyam erat sed diam
voluptua at vero eos et accusam et justo duo dolores et ea rebum stet clita
kasd gubergren no sea takimata sanctus est lorem ipsum dolor sit amet lorem
ipsum dolor sit amet consetetur sadipscing elitr sed diam nonumy eirmod
tempor invidunt ut labore et dolore magna aliquyam erat sed diam voluptua at
vero eos et accusam et justo duo dolores et ea rebum stet clita kasd
gubergren no sea takimata sanctus est lorem ipsum dolor sit amet lorem ipsum
dolor sit amet consetetur sadipscing elitr sed diam nonumy eirmod tempor
invidunt ut labore et dolore magna aliquyam erat sed diam voluptua at vero
eos et accusam et justo duo dolores et ea rebum stet clita kasd gubergren no
sea takimata sanctus est lorem ipsum dolor sit amet.

%----------------------------------------------------------------------------------
% Exemplo do ambiente para citações diretas com mais de 3 linhas.
%
% Para citações diretas de até 3 linhas, faça assim:
% De acordo com Direto~(2011, p.~21): "bla bla bla, bla 'bla'."
De acordo com Autor~(2011, p.~19):

\begin{directcite}
ut wisi enim ad minim veniam quis nostrud exerci tation ullamcorper suscipit
lobortis nisl ut aliquip ex ea commodo consequat duis autem vel eum iriure
dolor in hendrerit in vulputate velit esse molestie consequat vel illum
dolore eu feugiat nulla facilisis at vero eros et accumsan et
iusto odio
\end{directcite}

duis autem vel eum iriure dolor in hendrerit in vulputate velit esse
molestie consequat vel illum dolore eu feugiat nulla facilisis at vero eros
et accumsan et iusto odio dignissim qui blandit praesent luptatum zzril
delenit augue duis dolore te feugait nulla facilisi lorem ipsum dolor sit
amet consectetuer adipiscing elit sed diam nonummy nibh euismod tincidunt ut
laoreet dolore magna aliquam erat volutpat.

ut wisi enim ad minim veniam quis nostrud exerci tation ullamcorper suscipit
lobortis nisl ut aliquip ex ea commodo consequat duis autem vel eum iriure
dolor in hendrerit in vulputate velit esse molestie consequat vel illum
dolore eu feugiat nulla facilisis at vero eros et accumsan et iusto odio
dignissim qui blandit praesent luptatum zzril delenit augue duis dolore te
feugait nulla facilisi.

nam liber tempor cum soluta nobis eleifend option congue nihil imperdiet
doming id quod mazim placerat facer possim assum lorem ipsum dolor sit amet
consectetuer adipiscing elit sed diam nonummy nibh euismod tincidunt ut
laoreet dolore magna aliquam erat volutpat ut wisi enim ad minim veniam quis
nostrud exerci tation ullamcorper suscipit lobortis nisl ut aliquip ex ea
commodo consequat.

\include{_c-exemplo-cap1}

%----------------------------------------------------------------
% Aqui vai a bibliografia. Existem três estilos de citação: use
% 'ppgcc-alpha' para citações do tipo [Abc+] ou [XYZ] (em ordem
% alfabética na bibliografia), 'ppgcc-num' para citações
% numéricas do tipo [1], [20], etc., por ordem de citação, OU
% 'ppgcc-apalike' para citações do tipo "(Silva F., 2019) e 
% Silva F., (2019)".
%
%----------------------------------------------------------------
% NUMÉRICO:
\bibliographystyle{ppgcc-num}
%----------------------------------------------------------------
% ALPHA:
%\bibliographystyle{ppgcc-alpha}
%----------------------------------------------------------------
% APALIKE: 
% !!! Lembrar de fazer uso do pacote 'natbib'. Linha 70 !!!
% Funcionando em ambos os modos 'portuguese' e 'english'.
% Prefira esse (ppgcc-apalike) no lugar de somente 'apalike':
%\bibliographystyle{ppgcc-apalike}
%----------------------------------------------------------------
% !!! DICA APALIKE !!!:
% "Recomendamos o uso de dois estilos diferentes de citações para
% o APALIKE: 'citep' e 'citet'. O \citep traz os nomes dos autores 
% e também o ano para a referência, tudo entre parênteses 
% (utilizado nos finais de parágrafos). Enquanto que o '\citet' 
% atua como uma referência interna ao texto, por exemplo: 
% "De acordo com~\citet{borges:2020} o tema..." (deixando apenas 
% o ano entre parênteses e o nome dos autores fora dos parênteses). % E para as fontes de livros, os números das páginas 
% (opcionalmente) podem ser exibidos usando colchetes, com números 
% de páginas, como neste exemplo: 
% ~\citep[10-100]{CORMEMALGORITHMS01}."

%----------------------------------------------------------------
% Arquivo que consta as referências (.BIB)
\bibliography{_d-exemplo-bib}

%----------------------------------------------------------------
% Após \appendix, se iniciam os capítulos de Apêndice, com
% numeração alfabética.
%----------------------------------------------------------------
\appendix
\chapter{Meu primeiro apêndice}
\chapter{My second appendix}

%----------------------------------------------------------------
% Aqui vão os "capítulos" de anexos. Cada anexo deve
% ser considerado um capítulo.
%----------------------------------------------------------------
\anexos
\chapter{Meu primeiro anexo}
\chapter{My second attachment}

% E aqui (para a felicidade de todos) termina o documento.
\end{document}